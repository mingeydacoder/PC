\documentclass{article}

% Language setting
% Replace `english' with e.g. `spanish' to change the document language
\usepackage[english]{babel}

% Set page size and margins
% Replace `letterpaper' with `a4paper' for UK/EU standard size
\usepackage[letterpaper,top=2cm,bottom=2cm,left=3cm,right=3cm,marginparwidth=1.75cm]{geometry}
\usepackage{CJKutf8}
% Useful packages
\usepackage{amsmath}
\usepackage{graphicx}
\usepackage{setspace}
\usepackage{float}
\usepackage{subfigure}
\usepackage{array}
\usepackage[section]{placeins}
\usepackage[colorlinks=true, allcolors=blue]{hyperref}
\usepackage[export]{adjustbox}

\author{B10209040 陳彥倫}

\begin{document}
\thispagestyle{empty}
\hfill {\scshape \large Cloud Physics, Fall 2023 } \hfill {\scshape P1}
\smallskip
\hrule
\begin{CJK*}{UTF8}{bsmi}
\bigskip
\bigskip
\bigskip

\centerline{\huge \textbf {HW3}}
\bigskip
\centerline{\textbf {B10209040 陳彥倫}}

\section*{1.}

\begin{center}
    \begin{adjustbox}{width=150mm,center}
        \begin{tabular}{||c|c | c | c | c ||} 
            \hline
            Case & A & B & C & D \\ [1ex] 
            \hline\hline
            Time[sec] & 34 & 41 & 15 & 18 \\ [1ex] 
            \hline
            Radius[mm] & 1.08656 & 0.89265 & 2.48207 & 2.09009 \\ [1ex] 
            \hline
                    & +0.00156 & +0.00215 & +0.00027 & +0.00039 \\ [1ex] 
            \hline
        \end{tabular}
    \end{adjustbox}
\end{center}

\begin{spacing}{2}
    \begin{large}
        \: \\
        \textbf{Approach:} \\
        為求水滴從雲底墜落至地表所花時間,利用終端速度與顆粒半徑大小等關係式\\
        $\frac{dZ}{dt} = -u(R) + W$計算。然而在只有在雲中水滴會受到上升氣流之影響,因此將此處$W = 0$代入。
        半徑的遞迴計算以Mason equation求得。\\
        \textbf{Discussion:} \\
        由結果可以看出初始半徑較小的粒子到達地面所花費的時間較長,成長之半徑大小的比例亦較大。落至地面之最終半徑
        大小與預期不符,可能與單位的設定與公式計算有關。

        
        

        
    \end{large}
\end{spacing}

\end{CJK*}
\end{document}