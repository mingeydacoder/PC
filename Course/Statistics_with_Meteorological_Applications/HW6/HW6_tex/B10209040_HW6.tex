\documentclass{article}

% Language setting
% Replace `english' with e.g. `spanish' to change the document language
\usepackage[english]{babel}

% Set page size and margins
% Replace `letterpaper' with `a4paper' for UK/EU standard size
\usepackage[letterpaper,top=2cm,bottom=2cm,left=3cm,right=3cm,marginparwidth=1.75cm]{geometry}
\usepackage{CJKutf8}
% Useful packages
\usepackage{amsmath}
\usepackage{graphicx}
\usepackage{setspace}
\usepackage{float}
\usepackage{subfigure}
\usepackage[section]{placeins}
\usepackage[colorlinks=true, allcolors=blue]{hyperref}
\usepackage[export]{adjustbox}
\usepackage{array}

\author{B10209040 陳彥倫}

\begin{document}
\thispagestyle{empty}
\hfill {\scshape \large Statistics with Meteorological Applications, Spring 2024} \hfill {\scshape P1}
\smallskip
\hrule
\begin{CJK*}{UTF8}{bsmi}
\bigskip
\bigskip
\bigskip

\centerline{\huge \textbf {HW6}}
\bigskip
\centerline{\textbf {B10209040 陳彥倫}}


\section*{“Scientists rise up against statistical significance”}
    \begin{spacing}{2.5}
        \begin{large}
        這篇文章主要闡述了一個論點:捨棄所謂的統計顯著性差異。因許多科學家及統計學家都發現人們甚至是專業的研究領域人士有濫用及誤解
        「統計顯著性」的現象。即一般來說,統計上「差異不顯著」並不代表證明了虛無假設,「差異顯著」亦不代表其他假設是正確的。換句話說,
        我們不應該端看P value是否超過所定的門檻就做出資料有無差異或相關性的結論。而會造成此誤解的原因在於人類的認知傾向於對統計的結
        果做「分類」的動作,而這通常會導致分析者做出過度的高估及低估或其他誤導性的結果。因此作者等人提出的核心概念為提倡消除人們對P value、
        信賴區間等統計方法潛在的分類概念,強調不確定性避免有"over confidence"的情況發生。例如落在區間內的資料點表現出對母體的符合性
        ,並不代表區間外的就完全不相關,此時初始的統計假設就顯得相對重要。
        \end{large}
    \end{spacing}



\end{CJK*}
\end{document}