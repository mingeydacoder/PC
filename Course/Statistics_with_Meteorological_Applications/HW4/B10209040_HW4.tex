\documentclass{article}

% Language setting
% Replace `english' with e.g. `spanish' to change the document language
\usepackage[english]{babel}

% Set page size and margins
% Replace `letterpaper' with `a4paper' for UK/EU standard size
\usepackage[letterpaper,top=2cm,bottom=2cm,left=3cm,right=3cm,marginparwidth=1.75cm]{geometry}
\usepackage{CJKutf8}
% Useful packages
\usepackage{amsmath}
\usepackage{graphicx}
\usepackage{setspace}
\usepackage{float}
\usepackage{subfigure}
\usepackage[section]{placeins}
\usepackage[colorlinks=true, allcolors=blue]{hyperref}
\usepackage[export]{adjustbox}
\usepackage{array}

\author{B10209040 陳彥倫}

\begin{document}
\thispagestyle{empty}
\hfill {\scshape \large Statistics with Meteorological Applications, Spring 2024} \hfill {\scshape P1}
\smallskip
\hrule
\begin{CJK*}{UTF8}{bsmi}
\bigskip
\bigskip
\bigskip

\centerline{\huge \textbf {HW4}}
\bigskip
\centerline{\textbf {B10209040 陳彥倫}}

\section*{1.a.}
    \begin{spacing}{2}
        \begin{large}
            信賴區間是統計學中一個重要的概念,用於估計母體參數(例如平均值或比例)的不確定性。信賴區間提供了一個範圍,我們可以合理地認為包含了真實母體參數值的範圍。
            具體地,信賴區間由兩個數值組成,分別稱為下限和上限。這兩個數值通常是根據樣本統計量計算出來的,例如樣本平均值或樣本比例。信賴區間的計算通常基於樣本的分佈,以及對母體參數的估計和不確定性的理解。
            信賴區間的定義可以用以下方式描述:「如果我們重複地從同一個母體中抽取多個樣本,然後對每個樣本計算一個估計值(如樣本平均值),然後按照某種方法對這些估計值進行分析,例如計算它們的平均值和標準差,然後根據這些統計量,我們可以構建一個區間。這個區間被稱為信賴區間,表示我們對真實母體參數值的估計具有一定的信心水平。」
            信賴區間通常以一個信心水平(confidence level)來描述,例如95\%信心水平的信賴區間。這意味著在許多次抽樣中,我們可以期望有95\%的信心,真實母體參數值位於所計算的信賴區間中。常見的信心水平包括90\%、95\%和99\%等。
            信賴區間的計算方法取決於所研究的問題和樣本統計量。對於平均值的信賴區間,通常使用t分佈或正態分佈進行計算。對於比例的信賴區間,則使用二項分佈或正態分佈。\\
            \textbf{remark:}\\
            前半部的回覆概括了信賴區間在統計學上的概念及名詞解釋,而後提到信賴區間的定義通常以信賴水準來描述,與課堂上的理解相符。
            但對於t分布的使用並沒有詳細說明其通常是被用於母體變異數未知的情況。
        \end{large}
    \end{spacing}

\newpage

\thispagestyle{empty}
\hfill {\scshape \large Statistics with Meteorological Applications, Spring 2024} \hfill {\scshape P2}
\smallskip
\hrule
\bigskip
\bigskip
\bigskip

\section*{1.b.}
\begin{spacing}{2}
    \begin{large}
        不合理,信賴區間之相關定義不適合用機率的概念描述,母體參數的真實數值落在信賴區間內的結果只有0跟1的區別。應該改為在100次
        的抽樣統計當中,有95次此區間涵蓋了某人在這次大選的真實支持率。
    \end{large}
\end{spacing}

\section*{2.}
\begin{spacing}{2}
    \begin{large}
        此案例樣本數較大,且母體變異數未知,信賴區間的計算可套用:\\
        \begin{center}
            $\overline{X}\pm Z_{\alpha/2} \frac{S}{\sqrt{n}}$
        \end{center}
        而題幹所代入之數字符合上述公式,此敘述正確。又因其符合95\%信賴區間之定義,可以推斷[2.96,3.24]有0.95的機率包含
        真實平均。
    \end{large}
\end{spacing}

\section*{3.}
\begin{spacing}{2}
    \begin{large}
        因為母體平均數即為樣本分配的平均,計算可得此區間約為[235.26,244.74]。而236位於此區間內,因此機器運作良好。
    \end{large}
\end{spacing}

        
\newpage
\thispagestyle{empty}
\hfill {\scshape \large Statistics with Meteorological Applications, Spring 2024} \hfill {\scshape P3}
\smallskip
\hrule    


\end{CJK*}
\end{document}