\documentclass{article}

% Language setting
% Replace `english' with e.g. `spanish' to change the document language
\usepackage[english]{babel}

% Set page size and margins
% Replace `letterpaper' with `a4paper' for UK/EU standard size
\usepackage[letterpaper,top=2cm,bottom=2cm,left=3cm,right=3cm,marginparwidth=1.75cm]{geometry}
\usepackage{CJKutf8}
% Useful packages
\usepackage{amsmath}
\usepackage{graphicx}
\usepackage{setspace}
\usepackage{float}
\usepackage{subfigure}
\usepackage[section]{placeins}
\usepackage[colorlinks=true, allcolors=blue]{hyperref}
\usepackage[export]{adjustbox}
\usepackage{array}

\author{B10209040 陳彥倫}

\begin{document}
\thispagestyle{empty}
\hfill {\scshape \large Statistics with Meteorological Applications, Spring 2024} \hfill {\scshape P1}
\smallskip
\hrule
\begin{CJK*}{UTF8}{bsmi}
\bigskip
\bigskip
\bigskip

\centerline{\huge \textbf {HW5}}
\bigskip
\centerline{\textbf {B10209040 陳彥倫}}


\section*{1.}
    \begin{spacing}{2.5}
        \begin{large}
            虛無假設:2000年的平均降雨量與氣候平均無差異。\\
            對立假設:2000年的平均降雨量與氣候平均有差異。
        \end{large}
    \end{spacing}

\section*{2.}
    \begin{spacing}{2.5}
        \begin{large}
            虛無假設:近15年(1999-2013)的全球平均溫度不大於上世紀初15年(1901-1915)的平均溫度。\\
            對立假設:近15年(1999-2013)的全球平均溫度大於上世紀初15年(1901-1915)的平均溫度。\\
            將顯著水準設為0.05,意即Z=1.645,並按公式解出要滿足顯著大於之$\overline{X}$的門檻約為12.42°C
            而年平均處理後的近15年平均溫度約為12.08°C,顯示近15年的全球平均溫度並未顯著大於上世紀初15年的平均溫度。
        \end{large}
    \end{spacing}

\section*{3.}
\begin{spacing}{2.5}
    \begin{large}
        此案例樣本數較大,且母體變異數未知,信賴區間的計算可套用:\\
        \begin{center}
            $\overline{X}\pm Z_{\alpha/2} \frac{S}{\sqrt{n}}$
        \end{center}
        而題幹所代入之數字符合上述公式,此敘述正確。又因其符合95\%信賴區間之定義,可以推斷[2.96,3.24]有0.95的機率包含
        真實平均。
    \end{large}
\end{spacing}
        
\newpage
\thispagestyle{empty}
\hfill {\scshape \large Statistics with Meteorological Applications, Spring 2024} \hfill {\scshape P3}
\smallskip
\hrule    
\bigskip
\bigskip
\bigskip

\end{CJK*}
\end{document}